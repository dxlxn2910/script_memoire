%\documentclass[11pt]{beamer}
%\documentclass[aspectratio=169]{beamer}
\documentclass[handout]{beamer}
%\documentclass[10pt]{beamer}
\usepackage[utf8]{inputenc}
%\setbeamercolor{background canvas}{bg= green} % changer la couleur de l'arrière-plan
%\pgfpagesuselayout{2 on 1}[a4paper,border shrink=5mm]


\usepackage[T1]{fontenc}
%\usepackage{pgfpages}
%\usepackage{pslatex}
%\usepackage{mwe}
\usepackage{textcomp}
\usepackage{booktabs}
\usepackage{amsmath}
\usepackage[french]{babel}
%\usepackage{hyperref}
%\usepackage{fullpage}
\usepackage{amsfonts}
\usepackage{float}
%\usepackage[beamerthemesplit]{beamerthemesidebar}
\usepackage{caption}
\usepackage{subcaption}
\usepackage{amssymb}
\usepackage{tikz}
\usepackage{graphicx}
%\usepackage{bookman}
\usepackage{algorithm}
\usepackage{dsfont}
\usepackage{algorithmic}
\usepackage{lmodern}
%\usepackage[francais]{babel}
\usepackage{dsfont}
\usepackage{caption}
\usepackage{subcaption}
\usepackage{amssymb}
\usepackage{textcomp}
\usepackage{wasysym}
%\usepackage{fontspec}
%\usepackage{amsmath}
%\usepackage{amsfonts}
%\usepackage{float}
\floatname{algorithm}{Algorithme}
%\setmainfont{DejaVu Serif}

%\usetheme{Frankfurt} % Theme choice
%\usetheme[secheader]{Madrid}
%\usetheme{Warsaw}
\usetheme{AnnArbor}
%\usetheme{Berkeley}
%\usetheme{Luebeck}
%\usetheme{Malmoe}
%\usetheme{Berlin}
%\usetheme{Boadilla} 
%\usetheme{CambridgeUS}
%\usetheme{Copenhagen}
%\usetheme{Marburg}
%\usetheme{PaloAlto}

\usecolortheme{seahorse} % Color theme
%\usecolortheme{beaver}
%\usecolortheme{beetle}
%\usecolortheme{crane}
%\usecolortheme{dolphin}
%\usecolortheme{seagull}
%\usecolortheme{structure}

%\useinnertheme{circles}
%rtheme{}
\usefonttheme{professionalfonts}


%============mon chocoya======================================

\newtheorem{thm}{Théorème}[section]
\newtheorem{cor}[thm]{Corollaire}
\newtheorem{lem}[thm]{Lemme}
\newtheorem{pbm et hypo}[thm]{Problématique et hypothèses}
\newtheorem{notation}{Notation}[section]
\newtheorem{prop}[thm]{Proposition}
\newtheorem{prv}[thm]{Preuve}
\newtheorem{Conclusion}[thm]{Conclusion}
\newtheorem{Perspectives}[thm]{Perspectives}
\newtheorem{rem}[thm]{Remarque}
%\theoremstyle{defn}
\newtheorem{defn}[thm]{Définition}
\newtheorem{exam}[thm]{Exemple}


%=================Fin de mon chocoya=================================

%\usepackage{algorithm}
\setcounter{secnumdepth}{4}
%\usepackage{beamerthemeclassic}
\floatname{algorithm}{Algorithme}
%\usepackage{algorithmic}
\setbeamertemplate{headline}{}
%\setcounter{framenumber}{1}
%\setbeamertemplate{footline}{}
%\setbeamertemplate{background canvas}{\includegraphics[scale=0.5]{logouniv.jpg}}
%\setbeamertemplate{background canvas}[vertical shading][top=white,bottom=yellow]
%\setbeamercolor{background canvas}{bg=violet!10!white}
\hypersetup{pdfpagemode=FullScreen}
\beamerboxesdeclarecolorscheme{blocbleu}{blue}{yellow}
% Diapo 1 : Titre
\title[Feu $\Rightarrow$ Hybridation $\Rightarrow$ PINNs] %optional
{BIENVENUE Á LA SOUTENANCE DE MÉMOIRE DE MASTER}

%\subtitle{Demonstrating larger fonts}
\author[Présenté par : Dalan Coulibaly] % (optional)
{\textit{Présenté par} : Coulibaly Ladji Dalan Jean-Louis}
%\date{\today}

\institute[UPGC] % (optional)
{
	%	\inst{1}%
	{\large 	\textbf{Université Peleforo Gon Coulibaly}}\\
	{{\large \textbf{UFR Sciences Biologiques}}}\\
	{\normalsize \textbf{D\'{e}partement de Mathématiques-Physiques-Chimies}}\\
	{\normalsize	\textbf{OPTION : Mathématiques Fondamentales et appliquées}}
}
\date[\today]  
{\textit{Spécialité} : Analyse numérique}

% Use a simple TikZ graphic to show where the logo is positioned
%\logo{\includegraphics[height=1cm]{logoupgc.png}}

\begin{document}
	% Diapo 2 : tables des matieres
	
	
	\begin{frame}
		\begin{minipage}{\textwidth}
			\includegraphics[width=1.5cm]{D:/dxlxn/Master 2/Memoire/Code secret/upgc2-removebg-preview.png}%
			\hfill
			\includegraphics[width=2cm]{D:/dxlxn/Master 2/Memoire/Code secret/ufr-removebg-preview.png}
		\end{minipage}
		\vfill
		\titlepage
	\end{frame}
	
	\begin{frame}[plain] % 'plain' supprime l'entête et pied de page
		%	\usebackgroundtemplate{\includegraphics[width=\paperwidth,height=\paperheight]{exemple-revue-de-litterature.png}} 
		\begin{center}
			\vspace{2.5cm}
			\begin{beamercolorbox}[rounded=true, shadow=true, sep=1em, center]{title}
				{\huge \textbf{\underline{THÈME} : }}\\[0.5em]
				{\LARGE \textbf{RÉSOLUTION D'UN MODÈLE DE FEU DE VÉGÉTATION PAR LA MÉTHODE DES PINNs}}
			\end{beamercolorbox}
		\end{center}
	\end{frame}
	
	\begin{frame}
		\frametitle{Table des matières}
		\tableofcontents[sections=1-6]  % Affiche les sections 1 à 3
		%	\tableofcontents[pausesections]
	\end{frame}
	
	%\begin{frame}
	%	\frametitle{Table des matières (2/2)}
	%	\tableofcontents[sections=6-10]  % Affiche les sections 4 à 6
	%\end{frame}
	% Diapo 3 : contexte
	%diap1
	\begin{frame}[plain]
		\noindent
		\includegraphics[width=\paperwidth, height=\paperheight]{D:/dxlxn/Master 2/Memoire/Code secret/pp1.png}
	\end{frame}
	
	
	\begin{frame}
		\frametitle{Introduction}
		\section{Introduction}
		\usebackgroundtemplate{\includegraphics[width=\paperwidth,height=\paperheight]{im10.png}} 
		\begin{itemize}
			\item[$\clubsuit$] \textbf{\textcolor{red}{Contexte} : } \pause
			\begin{itemize}
				\item Global Forest Watch, 2024.  \pause
				\item Contribution étatique, scientifique, limitation, carnage. \pause
				\item Comprendre, prédire, modèle de feu, \textbf{Koo \textit{et al., }(2005)}. \pause
				\item Modèle physique, vitesse de propagation, résolue RK4+optimisation, états discontinues.  
			\end{itemize}
		\end{itemize}
		\pause[4]
		\begin{center}
			\includegraphics[width=0.7\linewidth]{im10_beamer}
			\captionof{figure}{Propagation du feu dans un combustible thermiquement mince.}
		\end{center}
	\end{frame}
	\begin{frame}
		\frametitle{Introduction (suite)}
		\begin{itemize}
			\item[$\clubsuit$] \textbf{\textcolor{red}{Problématique} : } \pause
			\begin{itemize}
				\item RK4(stabilité et précision), limité, maillage et coût. \pause
				\item Limitation, alternative innovante, introduit par \textbf{Raissi \textit{et al.,} (2019)}, approximateur. \pause
			\end{itemize}
		\end{itemize}
		\begin{itemize}
			\item[$\clubsuit$] \textbf{\textcolor{red}{Objectif} : } \pause
			\begin{itemize}
				\item Analyse mathématique du modèle. \pause
				\item Construction RN, prédiction de $R$, estimation de la température \\ et de l'humidité en fonction de $y$.
			\end{itemize}
		\end{itemize}
	\end{frame}
	%=================================part ===============================
	\begin{frame}[plain]
		\noindent
		\includegraphics[width=\paperwidth, height=\paperheight]{D:/dxlxn/Master 2/Memoire/Code secret/pp2.png}
	\end{frame}
	%1
	
	%2
	\begin{frame}
		\frametitle{Modèle de référence — Koo \textit{et al.} (2005)}
		\section{Généralité}
		\begin{itemize}
			\item[$\spadesuit$] \textbf{\textcolor{red}{Nature et objectif du modèle} : } \pause 
			\begin{itemize}
				\item Modèle physique basé sur la \textbf{conservation de l'énergie}. \pause
				\item Prédire la \textbf{vitesse de propagation} $R$. \pause
				\item Estimation l'évolution de la \textbf{température $T(y)$} et de l'\textbf{humidité $M_w(y)$} en fonction de $y$ fixé à l'origine. \pause
				
				\vspace{1ex}
				\item Hypothèses simplificatrices. \pause %car meme sans notion physqiue  de base, on sait tous que le feu n'est pas un jeu, a plus forte raison la modeliser. 
				
				\pause[2]
				\begin{center}
					\includegraphics[width=0.7\linewidth]{im0}
					\captionof{figure}{Transferts thermiques dans un élément de combustible.}
				\end{center}
				
				
			\end{itemize}
		\end{itemize}
	\end{frame}
	%
	\begin{frame}
		\frametitle{Formulation physique}
		\begin{itemize}
			\item[$\spadesuit$] \textbf{\textcolor{red}{Conservation de l'énergie} : } \pause 
			\begin{itemize}
				\item Parmi ces nombreuses hypothèses on néglige la conduction. \pause %qui est le transfert de chaleur dans le sol ici est négligeable. 
				\begin{block}{Équation du bilan énergétique}
					\vspace{-1.5ex}
					\begin{equation}\label{f:conservation}
						q_{\text{sensible}} + q_{\text{latent}} = q_{\text{rs}}  + q_{\text{ri}} + q_{\text{pr}}+ q_{\text{cs}} + q_{\text{ci}}
					\end{equation}
				\end{block}\pause
				
				\item \textbf{Gauche} : énergie absorbée par l'élément combustible (chauffage + évaporation). \pause
				\item \textbf{Droite} : contributions thermiques extérieures (radiation + convection). \pause
				
				\vspace{2.5ex}
				\begin{center}
					\item [{\Large \clock}]  \textcolor{red}{Attention}. 
				\end{center}
			\end{itemize}
		\end{itemize}
	\end{frame}
	
	\begin{frame}
		\frametitle{Énergie absorbée}
		\begin{itemize}
			\item[$\spadesuit$] \textbf{\textcolor{red}{Chaleur sensible} : } \pause 
			\begin{block}{Chaleur sensible}
				\vspace{-1.5ex}
				\begin{equation}
					q_{\text{sensible}} = 
					\begin{cases}
						-\rho_f C_{pf} R \phi \dfrac{dT}{dy}, & \text{si } T \neq 373\,\mathrm{K}, \\
						0, & \text{si } T = 373\,\mathrm{K}
					\end{cases}
				\end{equation}
			\end{block}
			
			\begin{itemize}
				\item $\rho_f$ : densité du combustible
				\item $C_{pf}$ : capacité thermique spécifique
				\item $\phi$ : compactage du lit
				\item $R$ : vitesse de propagation
				\item $T(y)$ : température à la position $y$
			\end{itemize}
		\end{itemize}
		
		\vspace{1ex}
		\small Chaleur utilisée pour élever la température jusqu'à la température d'inflammation $\approx 561\,\mathrm{K}$.
	\end{frame}
	\begin{frame}
		\frametitle{Énergie absorbée(Suite)}
		
		\begin{itemize}
			\item[$\spadesuit$] \textbf{\textcolor{red}{Chaleur latente} : } \pause 
			\begin{block}{Chaleur latente}
				\vspace{-1.5ex}
				\begin{equation}
					q_{\text{latent}} = 
					\begin{cases}
						\rho_f h_{\text{vap}} R \phi \dfrac{dM_w}{dy}, & \text{si } T = 373\,\mathrm{K}, \\
						0, & \text{si } T \neq 373\,\mathrm{K}
					\end{cases}
				\end{equation}
			\end{block}
			
			\begin{itemize}
				\item $h_{\text{vap}}$ : enthalpie de vaporisation
				\item $M_w(y)$ : humidité du combustible en $y$
			\end{itemize}
		\end{itemize}
		
		\vspace{1ex}
		\small Chaleur nécessaire pour évaporer l'humidité quand  $T = 373\,\mathrm{K}$.
	\end{frame}
	\begin{frame}
		\frametitle{Transferts de chaleur dans le combustible \phone}
		
		\begin{itemize}
			\item Deux mécanismes majeurs de transfert thermique :
			\begin{itemize}
				\item \textbf{Radiation} : émission de chaleur par la flamme.
				\item \textbf{Convection} : échange thermique entre le lit et l'air.
			\end{itemize}
			\item Plusieurs termes dans le modèle : \\
			$q_{\text{rs}}, q_{\text{ri}}, q_{\text{pr}}$ (radiatifs) et $q_{\text{cs}}, q_{\text{ci}}$ (convectifs)
		\end{itemize}
		
		\vspace{1ex}
		\begin{center}
			\includegraphics[width=0.55\linewidth]{im6}
			\captionof{figure}{Configurations de vent et pente (source : Koo et al., 2005)}
		\end{center}
	\end{frame}
	\begin{frame}
		\frametitle{Rayonnement thermique \phone}
		
		\begin{block}{Radiation de la flamme (surface du lit)}
			\vspace{-1.5ex}
			\begin{equation}
				q_{\text{rs}} = \frac{a_{fb}E_{fl}}{2l_f}\left[1 - \frac{Z}{\sqrt{1+Z^2}}\right] \tanh\left[\frac{2}{3}\left(\frac{w}{L_{fl}}\right)^{1/3}\right]
			\end{equation}
		\end{block}
		
		\begin{itemize}
			\item $E_{fl} = \epsilon_{fl} \sigma T_{fl}^4$ : puissance émissive de la flamme
			\item $Z$ : facteur géométrique de vue
			\item $\epsilon_{fl} = 1 - e^{-0.6L_{fl}}$
		\end{itemize}
		\textbf{Autres rayonnements} :
		\begin{itemize}
			\item \textcolor{blue}{Radiation interne} : $q_{\text{ri}} = 0.25sE_b e^{-0.25sy}$
			\item \textcolor{blue}{Perte radiative du lit} : $q_{\text{pr}} = -\frac{\epsilon_{fb} \sigma (T^4 - T_\infty^4)}{l_f}$
		\end{itemize}
	\end{frame}
	\begin{frame}
		\frametitle{Convection thermique \phone}
		
		\begin{block}{Convection de surface (selon vent/pente)}
			\vspace{-1ex}
			\begin{equation}
				q_{\text{sc}}= \frac{0.565k_{fl}\mathrm{Re}_y^{1/2} \mathrm{Pr}^{1/2}}{y l_f}\left(T_{g}-T(y)\right)e^{-\frac{0.3y}{L_{fl}}}
			\end{equation}
		\end{block}
		\vspace{3ex}
		\begin{block}{Convection interne (dans le lit)}
			\vspace{-1ex}
			\begin{equation}
				q_{\text{ic}} = \frac{0.911sk_b \mathrm{Re}_D^{0.385}\mathrm{Pr}^{1/3}}{D}\left(T_g-T(y)\right)e^{-0.25sy}
			\end{equation}
		\end{block}
		\vspace{3ex}
		\begin{itemize}
			\item $T_g$ : température de gaz ($T_{fl}$, $T_b$ ou $T_\infty$ selon le cas)
			\item Cas heading vs backing $\Rightarrow$ expression adaptée (voir mémoire)
		\end{itemize}
	\end{frame}
	
	\begin{frame}
		\frametitle{Résolution RK4 du modèle}
		
		\textbf{Koo \textit{et al.} (2005)}, en considérant $R$ comme une valeur propre à estimer par une procédure \textbf{d'optimisation} implicite.\pause
		
		\pause
		\begin{itemize}
			\item Le système est traité comme un système \textbf{à états discontinus par morceaux}.
			\item La vitesse de propagation obtenue est : \textbf{$R \approx 0{,}062$}.
			\item Les profils de la température $T(y)$ et de l'humidité obtenues sont : \pause 
		\end{itemize}
		
		\pause
		\begin{center}
			\includegraphics[width=0.85\linewidth]{im1 (2)}
			\captionof{figure}{Profils $T(y)$, $M_w(y)$ et contribution des flux (RK4)}
		\end{center}
	\end{frame}
	\begin{frame}
		\frametitle{Transition vers la Méthodologie \aquarius}
		Dans le but d'atteindre pleinement nos objectifs \pause
		\begin{itemize}
			\item[\maltese] A reformuler le modèle physique sous la forme d’un système hybride. \pause
			\begin{itemize}
				\item SP1 modélise l'augmentation de la température. 
				\item SP2 évaporation de l'humidité. 
			\end{itemize} \pause 
			\item[\maltese] Faire la mise en place d'un réseau de neurones. \pause
			\begin{itemize}
				\item Prédire la vitesse de Propagation $R$. 
				\item Estimer les profils de $T$ et de $M_w$. 
			\end{itemize}
		\end{itemize}
	\end{frame}
	
	\begin{frame}[plain]
		\noindent
		\includegraphics[width=\paperwidth, height=\paperheight]{D:/dxlxn/Master 2/Memoire/Code secret/pp3.png}
	\end{frame}
	
	%pour le moment 
	
	\begin{frame}
		\frametitle{Formulation mathématique du modèle}
		
		\begin{itemize}
			\item $y \in [0,a]$ : distance flamme-combustible.
			\item $T(y)$ : température locale,\quad $M_w(y)$ : humidité locale.
			\item $R$ : vitesse de propagation (à déterminer).
			\item $Q_1(y,T)$ : somme des flux de chaleur hors évaporation.
			\item $Q_2(y) = Q_1(y, 373)$ : flux constant pendant l'évaporation.
			\item Conditions initiales : 
			\[
			T(a) = T_a,\quad M_w(a) = M_a
			\]
		\end{itemize}
	\end{frame}
	
	\begin{frame}
		\frametitle{Systèmes SP1 et SP2}
		
		\begin{minipage}{0.48\linewidth}
			\textbf{SP1 (hors évaporation)} :
			\begin{equation*}
				\begin{cases}
					\dfrac{dT}{dy} = -\dfrac{Q_1(y,T)}{\gamma_1 R}, & T \neq 373 \\[0.4em]
					\dfrac{dT}{dy} = 0, & T = 373 \\[0.4em]
					T(a) = T_a
				\end{cases}
			\end{equation*}
		\end{minipage}
		\hfill
		\begin{minipage}{0.48\linewidth}
			\textbf{SP2 (évaporation)} :
			\begin{equation*}
				\begin{cases}
					\dfrac{dM_w}{dy} = \dfrac{Q_2(y)}{\gamma_2 R}, & T = 373 \\[0.4em]
					\dfrac{dM_w}{dy} = 0, & T \neq 373 \\[0.4em]
					M_w(a) = M_a
				\end{cases}
			\end{equation*}
		\end{minipage}
		
		\vspace{1.5em}
		\begin{itemize}
			\item SP1 modélise la variation de la température $T$ en fonction de $y$ (régime 1). 
			\item SP2 modélise l'évaporation de l'humidité $M_w$ en fonction de $y$ (régime 2).
		\end{itemize}
	\end{frame}
	
	\begin{frame}
		\frametitle{Expression des flux thermiques $Q_1(y,T)$}
		
		\begin{itemize}
			\item \textbf{Cas heading} (vent dans le sens de la flamme) :
			\begin{equation*}
				Q_1(y,T) = \alpha_1 \left(1 - \frac{ya - b}{\sqrt{1 + (ya - b)^2}} \right)
				+ \alpha_2 e^{-ry} + \alpha_3 (T^4 - T_\infty^4)
			\end{equation*}
			\vspace{-1em}
			\begin{equation*}
				\quad + \frac{\alpha_4}{\sqrt{y}} (T_{\text{fl}} - T) e^{-cy}
				+ \alpha_6 (T_b - T) e^{-ry}
			\end{equation*}
			
			\item \textbf{Cas backing} (vent opposé) :
			\begin{equation*}
				Q_1(y,T(y)) = \alpha_1 \left(1-\frac{ya-b}{\sqrt{1+(ya-b)^2}}\right) + \alpha_2e^{-ry}
				+ \alpha_3\left(T^4(y) - T_\infty^4\right)
			\end{equation*}
			\vspace{-1em}
			\begin{equation*}
				\quad + \left(\frac{\alpha_5}{(L_{fb}-y)^{1/2}}+ \alpha_7\right) 
				\left(T_\infty - T(y)\right)
			\end{equation*}
			
			\begin{center}
				\item[$\Rightarrow$] Dans la suite, on se place dans le cas \textbf{heading}.
			\end{center}
		\end{itemize}
	\end{frame}
	
	%
	
	\begin{frame}
		\frametitle{Analyse théorique du modèle}
		\section{Méthodologie}
		\begin{itemize}
			\item[\maltese] Étude de l'existence de solution.
			\begin{itemize}
				\item \textbf{Cauchy-Lipschitz} (1918) : garantit une solution \textit{unique} si $f$ est lipschitzienne.
				\item \textbf{Carathéodory} (1988) : solution existe même si $f$ est \textit{discontinue}, sous certaines conditions.
			\end{itemize}
			
			\pause
			\textbf{Systèmes discontinus : inclusion de Filippov}
			\begin{itemize}
				\item Approche pour traiter les discontinuités non couvertes par les deux théorèmes précédents.
				\item Utilise une \textit{inclusion différentielle} avec enveloppe convexe.
				\item Solution existe si $f$ est bornée et la discontinuité est sur une surface $C^1$.
			\end{itemize}
		\end{itemize}
	\end{frame}
	
	\begin{frame}
		\frametitle{Résolution analytique : cas régulier}
		\begin{itemize}
			\item[\maltese] \textbf{Pourquoi chercher une solution analytique ?}
			\begin{itemize}
				\item Comprendre le comportement global du système
				\item Vérifier la cohérence des résultats numériques
				\item Obtenir des garanties théoriques (existence, unicité)
			\end{itemize}
			
			\pause
			\item[\maltese] \textbf{Méthodes classiques (EDO régulières)}
			\begin{itemize}
				\item \textbf{Séparation des variables} : $b(y)y' = a(t)$
				\item \textbf{Facteur intégrant} : $y' + p(t)y = q(t)$
			\end{itemize}
		\end{itemize}
	\end{frame}
	
	\begin{frame}
		\frametitle{Résolution analytique : cas hybride}
		\begin{itemize}
			\item[\maltese] \textbf{Difficulté :} 
			\begin{itemize}
				\item Discontinuités $\Rightarrow$ non applicabilité des méthodes classiques globales
			\end{itemize}
			
			\pause
			\item[\maltese] \textbf{Méthode par morceaux}
			\begin{itemize}
				\item Découpage en intervalles de régularité $I_k$
				\item Résolution sur chaque $I_k$ séparément
				\item Raccordement aux points de transition :\\
				$y(t_{k+1}^+) = R(y(t_{k+1}^-))$
			\end{itemize}
			
			\pause
			\item[\maltese] \textbf{Limite :} nécessite la connaissance exacte des instants de transition
		\end{itemize}
	\end{frame}
	
	%1 ajoutés
	%2ajoutés
	\begin{frame}
		\frametitle{Méthodes numériques classiques et choix de RK4}
		\begin{itemize}
			\item[\maltese] Les méthodes classiques (Euler, RK2, RK3) montrent rapidement leurs limites :
			\begin{itemize}
				\item Sensibilité aux discontinuités
				\item Instabilité si le champ vectoriel n'est pas lisse
			\end{itemize}
			
			\pause
			\item[\maltese] \textbf{Runge-Kutta d'ordre 4 (RK4)} : 
			\begin{itemize}
				\item Bonne stabilité et précision sur les EDO régulières
				\item Appliquée avec succès dans \textbf{Koo et al. (2005)} pour $R = 0.062$
				\item Couplée à une méthode de tir $\Rightarrow$ résultats satisfaisants
				\item Mais : nécessite une gestion manuelle des discontinuités ($T = 373$ K)
			\end{itemize}
		\end{itemize}
	\end{frame}
	
	%3ajoutées
	%\begin{frame}
	%	\frametitle{Méthodes numériques adaptées aux systèmes hybrides}
	%	\begin{itemize}
		%		\item[\maltese] \textbf{Event-Driven :} détecte précisément les transitions $g(y) = 0$.
		%		\item[\maltese] \textbf{Méthode pénalisée :} remplace les sauts par des fonctions douces (ex: tanh).
		%		\item[\maltese] \textbf{Méthodes "if-else" :} introduisent la logique dans l'algorithme classique (réduction automatique de $h$).
		%		\item[\maltese] \textbf{Problèmes :} surcharge algorithmique, difficulté d'adaptation en cas de multiples régimes.
		%	\end{itemize}
	%\end{frame}
	%4ajoutées 
	\begin{frame}
		\frametitle{Perspectives : limites et alternatives}
		\begin{itemize}
			\item[\maltese] Les méthodes classiques nécessitent des ajustements lourds dans les systèmes hybrides.
			\item[\maltese] Détection automatique des régimes reste difficile.
			\item[\maltese] Fortes non-linéarités limitent la stabilité et la précision.
		\end{itemize}
		\begin{alertblock}{Perspectives}
			Les PINNs émergent comme alternative prometteuse : intégration directe de la physique, flexibilité pour gérer les transitions.
		\end{alertblock}
	\end{frame}
	%&revoir 
	
	%RN
	
	\begin{frame}
		\frametitle{Principe et structure général des PINNs}
		\pause
		\begin{block}{Principe des PINNs}
			Au lieu d'intégrer numériquement les équations, on entraîne un réseau de neurones à approximer la solution, en imposant le respect des lois physiques comme contraintes d'apprentissage.
		\end{block}
		
		\textbf{Structure des PINNs}
		\begin{columns}
			\begin{column}{0.5\textwidth}
				\begin{itemize}
					\item[•] \textbf{Couche d'entrée} : Variables indépendantes ($x_1, x_2$).
					\item[•] \textbf{Couches cachées} : 
					\begin{itemize}
						\item Transformations non-linéaires
						\item Fonctions d'activation (tanh, ReLU)
					\end{itemize}
					\item[•] \textbf{Couche de sortie} : Solution prédite $\hat{u}(t,\theta)$
				\end{itemize}
			\end{column}
			\begin{column}{0.5\textwidth}
				\centering
				\includegraphics[width=\textwidth]{im17} % À remplacer par votre figure
				\footnotesize\textit{Architecture général d'un PINN}
			\end{column}
		\end{columns}
	\end{frame}
	%ici
	\begin{frame}
		\frametitle{PINNs pour les Systèmes Hybrides}
		\textbf{Rappel Système Hybride :}
		\begin{itemize}
			\item[$\divideontimes$] \textcolor{blue}{Régime 1 (SP1)} : 
			$\frac{dT}{dy} = -\frac{Q_1(y,T)}{\gamma_1 R}$  \hfill (si $T \neq 373$ K)
			\item[$\divideontimes$] \textcolor{red}{Régime 2 (SP2)} : 
			$\frac{dM_w}{dy} = \frac{Q_2(y)}{\gamma_2 R}$  \hfill (si $T = 373$ K)
		\end{itemize}
		
		\begin{block}{Formulation au sens des PINNs}\pause
			%	Soient : \\
			\begin{itemize}
				\item[*] 	\textcolor{red}{$ \hat{T}(y, \theta_T), \hat{M}_w(y, \theta_M)$} et \textcolor{red}{$\hat{R}( \theta_R)$} les approximations % par réseau de neurones des fonctions $T(y), M_w(y)$, et du paramètre $R$. 
				\item[*] \textcolor{red}{$\theta = (\theta_T, \theta_M, \theta_R) $} l'ensemble des poids et biais. % optimisé pendant l'entraînement. % de sorte à satisfaire à la fois, les équations différentielles et les conditions initiales. 
			\end{itemize}
		\end{block}
		
		\textbf{Structure du Réseau :}
		\begin{itemize}
			\item[$\divideontimes$] Entrée ($y \in [0,12m]$) $\Rightarrow$ nn.Linear(1, 64), nn.Tanh(), 
			\item[$\divideontimes$] 2 couches cachées nn.Linear(64, 64), nn.Tanh(),
			\item[$\divideontimes$] Sortie ($\hat{T}$, $\hat{M}_w$, $\hat{R}$) $\Rightarrow$ nn.Linear(64, 1) 
		\end{itemize}
	\end{frame}
	
	\begin{frame}
		\frametitle{Gestion de la transition, du paramètre $\hat{R}$ \& optimisation}
		\begin{itemize}
			\item[\maltese] Gestion de la transition.\pause
			\begin{itemize}
				\item Discontinuité en $T=373\,\mathrm{K}$.
				\item Fonction de transition $\mathcal{X}(T)$ autour de $373\,\mathrm{K}$ tel que :
				\begin{equation}
					\mathcal{X}(T) = \sigma\left(\lambda\left(T-373\right)\right) \text{ avec } \lambda >> 1
				\end{equation}
			\end{itemize}
			\item[\maltese] Gestion du paramètre $\hat{R}$.\pause
			\begin{itemize}
				\item Régularisation de $ \hat{R}$ sur $\left[R_{\text{min}}, R_{\text{max}}\right]$ : 
				\begin{equation}
					\hat{R}(\theta_R) = R_{\text{min}} + (R_{\text{max}} -  R_{\text{min}})\cdot\sigma\left(z(\theta_R)\right)
				\end{equation}
			\end{itemize}
			\item[\maltese]  Optimisation :
			\begin{itemize}
				\item Phase 1 : Adam (gradients bruités)
				\item Phase 2 : L-BFGS (précision)
			\end{itemize}
		\end{itemize}
	\end{frame}
	
	
	\begin{frame}
		\frametitle{Formulation des résidus \& Loss}
		\begin{block}{Résidus de l'EDO}\pause
			\vspace{-1ex}
			\begin{itemize}
				\item Pour SP1 : 	$\mathcal{R}_T(y) = \left(1-\mathcal{X}(T)\right)\cdot\left(\frac{d\hat{T}}{dy} + \frac{Q_1(y, \hat{T})}{\gamma_1 \hat{R}}\right)$.\pause
				\item Pour SP2 :$ \mathcal{R}_{M_w}(y) = \mathcal{X}(T)\cdot\left(\frac{d\hat{M_w}}{dy} - \frac{Q_2(y)}{\gamma_2 \hat{R}}\right)$.
			\end{itemize}\pause
		\end{block}
		
		\begin{block}{Résidus des Conditions initiales}\pause
			\vspace{-1ex}
			\begin{itemize}
				\item Pour SP1 : $\mathcal{C}_T(y_0) = \hat{T}(a) - T_{a}$.\pause
				\item Pour SP2 :$\mathcal{C}_{M_w} (y_0) = \hat{M}_w(a) - M_{a}  $.
			\end{itemize}\pause
		\end{block}
		
		\begin{block}{Fonction de perte global (Loss)} \pause 
			\vspace{-3 ex}
			\begin{align}
				\mathcal{L}(\theta) = \mathcal{L}_{\text{EDO}} + \mathcal{L}_{\text{CI}}, 
			\end{align}\pause
			Avec : $ \mathcal{L}_{\text{EDO}}= \frac{1}{N}\sum_{i=1}^{N}\left[\mathcal{R}_T(y_i)^2 +  \mathcal{R}_{M_w}(y_i)^2\right]$
		\end{block}
	\end{frame}
	
	
	
	% essai 
	
	
	\begin{frame}
		\frametitle{Imposition des conditions}
		\begin{itemize}
			\item[\maltese] Imposition pas la pénalisation.\pause			
			\begin{align}
				\mathcal{L}( \theta ) = \mathcal{L}_{\text{EDO}} + \lambda\mathcal{L}_{\text{CI}}
			\end{align}
			\small\[
			\mathcal{L}(\theta) = \underbrace{\frac{1}{N}\sum_i \mathcal{R}_T(y_i)^2 + \mathcal{R}_{M_w}(y_i)^2}_{\text{Résidus EDO}} + \lambda_1 \underbrace{(\hat{T}(a)-T_a)^2 + \lambda_2 (\hat{M}_w(a)-M_a)^2}_{\text{Conditions Initiales}}
			\]
			\item[\maltese]  Imposition par la transformation exacte. \pause
			\begin{equation}
				\mathcal{L}( \theta ) = \mathcal{L}_{\text{EDO}}
			\end{equation}
			
			{ Avec } $\hat{T}(y) = T_{a} + (y-a) \cdot\mathcal{N}_T(y; \theta_T)$  \\
			\vspace{1ex}
			et $ \hat{M_w}(y) = M_{a}+ (y-a)\cdot\mathcal{N}_M(y; \theta_M)$
			%			Où $\mathcal{N}_T \in [0, 1]$ et $\mathcal{N}_M \in [0, 1]$ sont des sorties du réseau entièrement connectés, mais nuls aux bornes. Ce qui permet d'imposer exactement que $T_{a} = T_{a}$ et  $M_w{(a)} = M_{a}$
		\end{itemize}
	\end{frame}
	
	%voir 
	\begin{frame}
		\frametitle{Cas d'Étude : Heading}
		\begin{itemize}
			\item[\maltese] Paramètres Clés :
			\begin{table}[h]
				\scriptsize
				\begin{tabular}{@{}ll@{}}
					\toprule
					\textbf{Paramètre} & \textbf{Valeur} \\
					\midrule
					Longueur $Y$ & 12 m \\
					Pente $\Omega_s$ & +17\textdegree \\
					Vitesse vent $U_w$ & +1.1 m/s \\
					Temp. ignition $T_{a}$ & 303 K \\
					Fraction eau $M_w(a)$ & 0.11 \\
					\bottomrule
				\end{tabular}
			\end{table}
			
			\vspace{0.2cm}
			\item[\maltese] Données d'étude 
			\begin{itemize}
				\item Les données du bouleau blanc de \textbf{Koo \textit{et al.,} (2005)}.
				\item Les données supposés.
			\end{itemize} 
			
			\vspace{0.2cm}
			\item[\maltese] Approche : 
			\begin{itemize}
				\item[•] Pénalisation des CI ($\lambda_1 = 90 000 \text{ et } \lambda_2 = 10^{20}$)
				\item[•] 5 000 points de collocation
			\end{itemize}
		\end{itemize}
	\end{frame}
	
	
	\begin{frame}[plain]
		\noindent
		\includegraphics[width=\paperwidth, height=\paperheight]{D:/dxlxn/Master 2/Memoire/Code secret/pp4.png}
	\end{frame}
	
	\begin{frame}
		\frametitle{Résultats de l'analyse complète du modèle}
		\framesubtitle{Synthèse avant l'implémentation par les PINNs}
		\section{Résultats \& Discussion}
		
		\begin{itemize}
			\item[\maltese] \textbf{Nature du système :} Système hybride d'EDO à transition conditionnelle activée par $T = 373\,\mathrm{K}$ entre deux régimes SP1/SP2.
			
			\item[\maltese] \textbf{Existence de solution :} Assurée globalement au sens de \textit{Filippov}, malgré la discontinuité de la dérivée en $T = 373$.
			
			\item[\maltese] \textbf{Échec de la résolution analytique :} Complexité de $Q_1(y,T)$ et absence des points $y_1$, $y_2$ empêche toute solution explicite par morceaux.
			
			\item[\maltese] \textbf{Limites des méthodes numériques classiques :}  
			Méthodes d'Euler, RK2 et RK3 inadaptées ; seule la méthode \textbf{RK4 + optimisation} (cf. \textit{Koo et al.}) donne des résultats fiables.
		\end{itemize}
		
		%	"Avant de passer à l’implémentation des PINNs, nous avons effectué une analyse complète du modèle.
		%	D'abord, il s'agit d’un système hybride à transition conditionnelle, donc avec des changements de régime selon la température.
		%	Ensuite, la discontinuité à $T = 373,\mathrm{K}$ exclut les théorèmes classiques, mais une solution globale est assurée par l'inclusion de Filippov.
		%	Côté analytique, même en linéarisant les flux, on ne peut pas expliciter les intégrales ou déterminer les points de transition $y_1$ et $y_2$, donc pas de solution fermée.
		%	Enfin, les méthodes numériques standards échouent à cause du saut brutal. Seule la méthode RK4 combinée à une procédure de tir permet de retrouver des résultats cohérents, mais elle ne détermine pas $R$ automatiquement."
	\end{frame}
	
	\begin{frame}
		\frametitle{Résultats - Profils séparés de $T(y)$ et $M_w(y)$}
		
		\textbf{Idée clé :} 
		\begin{itemize}
			\item $T$ croît lentement jusqu’à $373\,\mathrm{K}$, puis plus vite jusqu'à $556.6\,\mathrm{K}$
			\item $M_w$ reste stable puis décroît rapidement après évaporation
		\end{itemize}
		
		\pause
		\begin{center}
			\includegraphics[width=0.85\linewidth]{py3}
			\captionof{figure}{Profils séparés de la température $T$ et de l'humidité $M_w$ en fonction de $y$}
		\end{center}
		
		%	Avant de cliquer :
		%	
		%	"Nous allons maintenant examiner les résultats obtenus par les PINNs, en commençant par les profils individuels de la température et de l’humidité en fonction de la distance au front de flamme."
		%	
		%	Une fois affichée :
		%	
		%	"On observe une montée progressive de la température jusqu'à 373 K, puis une accélération liée à l'assèchement du combustible. L'humidité, elle, reste stable au départ, puis chute brusquement une fois ce seuil atteint."
	\end{frame}
	
	
	\begin{frame}
		\frametitle{Résultats - Profils combinés de $T(y)$ et $M_w(y)$}
		
		\textbf{Idée clé :} 
		\begin{itemize}
			\item Transition visible autour de $373\,\mathrm{K}$
			\item Apprentissage simultané réussi des deux profils
		\end{itemize}
		
		\pause
		\begin{center}
			\includegraphics[width=0.85\linewidth]{py6}
			\captionof{figure}{Profils combinés de $T$ et $M_w$ selon la distance $y$}
		\end{center}
		%	Avant de cliquer :
		%	
		%	"Pour mieux visualiser le lien entre la température et l’humidité, voici les deux courbes superposées."
		%	
		%	Une fois affichée :
		%	
		%	"La transition est ici plus évidente. On voit que la décroissance de l'humidité coïncide précisément avec le seuil critique de température, ce qui valide bien le changement de régime SP1 vers SP2."
	\end{frame}
	
	
	\begin{frame}
		\frametitle{Résultats - Contribution des flux thermiques}
		
		\textbf{Idée clé :} 
		\begin{itemize}
			\item Flux surfaciques dominants : $q_{\text{rs}}, q_{\text{cs}}$
			\item Flux internes négligeables : $q_{\text{ri}}, q_{\text{ci}}$
		\end{itemize}
		
		\pause
		\begin{center}
			\includegraphics[width=0.80\linewidth]{py5}
			\captionof{figure}{Contributions respectives des flux thermiques en fonction de $y$}
		\end{center}
		%	Avant de cliquer :
		%	
		%	"Au-delà des profils, nous avons aussi analysé la contribution des flux thermiques dans le transfert d’énergie."
		%	
		%	Une fois affichée :
		%	
		%	"Le rayonnement surfacique domine nettement, suivi par la convection surfacique. Les flux internes sont négligeables, ce qui confirme que la majorité des échanges se font en surface du lit de combustible."
	\end{frame}
	
	
	\begin{frame}
		\frametitle{Résultats - Valeurs numériques des profils et flux}
		
		\textbf{Idée clé :} 
		\begin{itemize}
			\item Vitesse apprise : $R = 0.068$
			\item Profils cohérents avec les données physiques et simulation RK4
		\end{itemize}
		
		\begin{center}
			\begin{table}[h]
				\footnotesize
				\caption{Valeurs numériques des différents profils et flux selon $y$}
				\begin{tabular}{@{}lrrrrrrrr@{}}
					\toprule
					$y$ & $T$ & $M_w$ & $q_{rs}$ & $q_{cs}$ & $q_{ri}$ & $q_{ci}$ & $q_{pr}$ & $Q_1$ \\
					\midrule
					0.0  & 556.6 & 0.039 & 135.2 & 92.8 & 32.3 & 0.7 & -33.7 & 227.4 \\
					1.0  & 535.8 & 0.106 & 86.0 & 25.2 & 0.4 & 0.1 & -24.0 & 87.7 \\
					2.0  & 456.7 & 0.109 & 34.8 & 16.8 & 0.0 & 0.0 & -13.6 & 38.0 \\
					4.0  & 353.1 & 0.110 & 7.4 & 9.9 & 0.0 & 0.0 & -2.8 & 14.5 \\
					6.0  & 315.5 & 0.110 & 2.9 & 6.0 & 0.0 & 0.0 & -0.6 & 8.3 \\
					8.0  & 306.4 & 0.110 & 1.5 & 3.7 & 0.0 & 0.0 & -0.2 & 5.0 \\
					12.0 & 303.0 & 0.110 & 0.6 & 1.5 & 0.0 & 0.0 & -0.0 & 2.1 \\
					\bottomrule
				\end{tabular}
			\end{table}
		\end{center}
		%	Avant de cliquer :
		%	
		%	"Enfin, voici un tableau numérique qui résume l’évolution des différentes grandeurs physiques mesurées tout au long du profil."
		%	
		%	Une fois affichée :
		%	
		%	"On peut y retrouver les valeurs exactes de température, humidité, flux individuels et chaleur totale selon la position. À noter que la vitesse de propagation $R$ apprise automatiquement est d’environ 0.068, proche de celle obtenue par RK4."
	\end{frame}
	
	
	\begin{frame}
		\frametitle{Conclusion}
		\section{Conclusion}
		\begin{itemize}
			\item Cette étude à porté sur un modèle de feu  inspiré de Koo \textit{et al.} (2005).
			\item Reformulation, l'analyse théorique (existence de solution, resolution analytique et numérique).
			\item L'approche par \textbf{PINNs} a permis d'obtenir une solution continue et différentiable en lissant la discontinuité, tout en apprenant simultanément $T(y)$, $M_w(y)$ et $R$.
		\end{itemize}
		\vspace{0.2cm}
		Toutefois en dépit de la capacité des PINNs à capturer la dynamique du système, leur coût d'entraînement reste élevé et les performances restent inférieures aux méthodes classiques pour les problèmes directs.
		
		\vspace{0.4cm}
		\textbf{En résumé :} Cette étude valide l'applicabilité des PINNs pour des systèmes hybrides bien reformulés, sous réserve d’une bonne gestion des discontinuités.
		
	\end{frame}
	
	\begin{frame}
		\frametitle{Perspective}
		\section{Perspective}
		Au vu des difficultés rencontrées, pour égaler la précision des méthodes numériques classiques et les résultats obtenus par \textbf{Koo \textit{et al}., (2005)} il s'avère important de proposer les perspectives suivantes : 
		
		\begin{itemize}
			\item[$\maltese$] \textbf{Méthode de tir + PINNs} pour respecter exactement les conditions de référence. 
			\item[$ \maltese$]	Utiliser une méthode  numériques classiques permettant de capturer l'instant de chaque transition, avant appliquer les PINNs sur chaque régime individuellement. 
			\item[$\maltese$] \textbf{XPINNs (Extended PINNs)} : découper le domaine en sous-domaines avec des réseaux spécialisés, pour mieux gérer les transitions brutales et améliorer la stabilité de l'entraînement.
		\end{itemize}
	\end{frame}
	
	\begin{frame}
		\frametitle{Remerciements }
		
		% Police élégante (choisir selon ce qui est disponible)
		\fontfamily{phv}\selectfont % Helvetica
		% Alternatives: \usepackage{mathpazo} pour Palatino, ou \usepackage{arev} etc.
		
		\centering
		\vspace{1cm}
		\textcolor{blue!85!black}{\textbf{\Huge MERCI À TOUS}}\\
		\vspace{0.5cm}
		\textcolor{blue!75!black}{\textbf{\Huge POUR VOTRE}}\\
		\vspace{0.5cm}
		\textcolor{blue!65!black}{\textbf{\Huge AIMABLE ATTENTION !}}
	\end{frame}
	
	
	%===========================================================
	
\end{document}